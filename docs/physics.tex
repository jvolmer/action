\RequirePackage[l2tabu,orthodox]{nag} % turn on warnings because of bad style
\documentclass[a4paper,11pt,bibtotoc]{scrartcl}

% This allows to type UTF-8 characters like ä,ö,ü,ß
\usepackage[utf8]{inputenc}

\usepackage[T1]{fontenc}        % Tries to use Postscript Type 1 Fonts for better rendering
% \usepackage{lmodern}            % Provides the Latin Modern Font
% which offers more glyphs than the default Computer Modern
\usepackage[largesc]{newpxtext}
\usepackage{newpxmath}
\usepackage{titling, titlesec}
\usepackage{slantsc} 
\usepackage{amsmath} % Provides all mathematical commands

\usepackage{hyperref}           % Provides clickable links in the PDF-document for \ref
\hypersetup{%
    %draft, % = no hyperlinking at all (useful in b/w printouts)
    colorlinks=true, linktocpage=true, pdfstartpage=3, pdfstartview=FitV,%
    % uncomment the following line if you want to have black links (e.g., for printing)
    %colorlinks=false, linktocpage=false, pdfstartpage=3, pdfstartview=FitV, pdfborder={0 0 0},%
    breaklinks=true, pdfpagemode=UseNone, pageanchor=true, pdfpagemode=UseOutlines,%
    plainpages=false, bookmarksnumbered=true, bookmarksopen=true, bookmarksopenlevel=0,%
    hypertexnames=true, pdfhighlight=/O,%nesting=true,%frenchlinks,%
    urlcolor=blue, linkcolor=red, citecolor=webgreen, %pagecolor=RoyalBlue,%
}   


\usepackage{grffile}            % Allow you to include images (like graphicx). Usage: \includegraphics{path/to/file}

% Allows to set units
\usepackage[ugly]{units}        % Allows you to type units with correct spacing and font style. Usage: $\unit[100]{m}$ or $\unitfrac[100]{m}{s}$

% Additional packages
\usepackage{url}                % Lets you typeset urls. Usage: \url{http://...}
\usepackage{breakurl}           % Enables linebreaks for urls
\usepackage{xspace}             % Use \xpsace in macros to automatically insert space based on context. Usage: \newcommand{\es}{ESPResSo\xspace}
\usepackage{xcolor}             % Obviously colors. Usage: \color{red} Red text
\usepackage{booktabs}           % Nice rules for tables. Usage \begin{tabular}\toprule ... \midrule ... \bottomrule

% Source code listings
\usepackage{listings}           % Source Code Listings. Usage: \begin{lstlisting}...\end{lstlisting}
\lstloadlanguages{python}       % Default highlighting set to "python"

\setlength{\droptitle}{-4\baselineskip} % Move the title up
\pretitle{\begin{center}\Large\scshape\bfseries} % Article title formatting
\posttitle{\end{center}} % Article title closing formatting
\titleformat{\section}[block]{\large\scshape}{\thesection.}{0.5em}{}
                                                                 
\begin{document}

\title{Simulating the Ising Model}
\author{Julia Volmer}
\date{\today}
\maketitle

% \tableofcontents


\section{Model}
Simulation of a ferromagnet without an external field
2 dimensional spin field ("lattice") $\textbf{s}$ of size L x L, where $s_i \in {-1, 1}$ can only point upwards (+1) or downwards (-1).
Hamilton operator (why is not the action defined?)
$$H(\textbf{s}) = - \sum_{i,j} s_i s_j$$

\section{Monte Carlo Simulation}
We want to compute the expectation value of some observable O in a statistical canonical ensemble
\begin{equation}
<O> = \sum_{\textbf{s}} O(\textbf{s}) P(\textbf{s})
\end{equation},
where
\begin{equation}
P(\textbf{s}) = \frac{\exp(-H(\textbf{s})/T)}{\sum_{\textbf{s}} \exp(-H(\textbf{s})/T)}
\end{equation}
is the probability to find a system with energy $H$ with temperature $T$ (also called Boltzmann weight). Here one specific spin lattice $\textbf{s}$ is called "configuration".

With Monte Carlo we just choose configurations according to the Boltzmann weight $P$ such that the expectation value becomes
\begin{equation}
<O> = 1/N \sum_{n=1}^N O(\textbf{s}_n).
\end{equation}

How to choose the configurations according the the Boltzmann weight? Via a Markov chain, where we generate a new configuration \(\textbf{s}_{new}\) out of another one \(\textbf{s}\).
For this we generate a configuration candidate $\textbf{s}'$  which is just $\textbf{s}$ with one spin flipped. This candidate is accepted (becomes the new configuration) with transition probability
\begin{equation}
W(\textbf{s} -> \textbf{s}') = min (1, \exp(-\Delta E / T)),
\end{equation}
where \(\Delta E = E(\textbf{s}') - E(\textbf{s})\) is the energy difference between the two configurations.
If the candidate is not accepted, $\textbf{s}_{new} = \textbf{s}$. If $Delta E <= 0$, the candidate is always accepted, otherwise it is accepted with probability $\exp(-\Delta E / T)$ (which simulates quantum fluctuation and prevents to fall into local minima).

The system is local and takes only next neighbor interactions into account,  therefore $\Delta E = 4 s_i \sum_{s_j \in NN(s_i)} s_j$ for a candiate configuration where the spin $s_i$ is flipped and only only next neighbors of $s_j$ are taken into account.

\section{Magnetization}
Observable total magnetization is calculated via
\begin{equation}
  m = |\sum_i s_i|.
\end{equation}

\section{Periodic boundary conditions}
\section{Thermalization of lattice}
\section{One MC step}
give each spin possibility to flip to decrease autocorrelation.


\end{document}